\documentclass[11pt]{article}

\usepackage[fleqn]{amsmath}
\usepackage{hyperref}
\usepackage{etoolbox}
\let\bbordermatrix\bordermatrix
\patchcmd{\bbordermatrix}{8.75}{4.75}{}{}
\patchcmd{\bbordermatrix}{\left(}{\left[}{}{}
\patchcmd{\bbordermatrix}{\right)}{\right]}{}{}


\begin{document}

{
    \begin{center}
        \huge
        ESC103 Gaussian Elimination Assignment
    \end{center}

    \begin{center}
        \large
        Kevin (Zerui) Wang
    \end{center}

    \vspace{3ex}
}

\section{Solving with Gaussian Elimination}
Reducing to RREF:
\begin{align*}
& \begin{bmatrix}
    1 & 0 & -6 & 0 \\
    0 & 2 & -12 & 0 \\
    2 & 1 & -6 & -2
\end{bmatrix}\!\!
\begin{bmatrix}
    x_1 \\
    x_2 \\
    x_3 \\
    x_4
\end{bmatrix}  =
\begin{bmatrix}
    0 \\
    0 \\
    0
\end{bmatrix} \\
\implies & \left[\begin{array}{@{}cccc|c@{}}
    1 & 0 & -6 & 0 & 0\\
    0 & 2 & -12 & 0 & 0\\
    2 & 1 & -6 & -2 & 0
\end{array} \right]
\begin{array}{c}
    \\
    \\
    \!R3-2R1
\end{array} \\
\implies & \left[\begin{array}{@{}cccc|c@{}}
    1 & 0 & -6 & 0 & 0\\
    0 & 2 & -12 & 0 & 0\\
    0 & 1 & 6 & -2 & 0
\end{array} \right]
\begin{array}{c}
    \\
    \!R2/2\\
    \;
\end{array} \\
\implies & \left[\begin{array}{@{}cccc|c@{}}
    1 & 0 & -6 & 0 & 0\\
    0 & 1 & -6 & 0 & 0\\
    0 & 1 & 6 & -2 & 0
\end{array} \right]
\begin{array}{c}
    \\
    \!\\
    \! R3-R2
\end{array} \\
\implies & \left[\begin{array}{@{}cccc|c@{}}
    1 & 0 & -6 & 0 & 0\\
    0 & 1 & -6 & 0 & 0\\
    0 & 0 & 12 & -2 & 0
\end{array} \right]
\begin{array}{c}
    \\
    \!\\
    \! R3/12
\end{array} \\
\implies & \left[\begin{array}{@{}cccc|c@{}}
    1 & 0 & -6 & 0 & 0\\
    0 & 1 & -6 & 0 & 0\\
    0 & 0 & 1 & -\frac{1}{6} & 0
\end{array} \right]
\begin{array}{c}
    \\
    \!R2 + 6R3\\
    \!
\end{array} \\
\implies & \left[\begin{array}{@{}cccc|c@{}}
    1 & 0 & -6 & 0 & 0\\
    0 & 1 & 0 & -1 & 0\\
    0 & 0 & 1 & -\frac{1}{6} & 0
\end{array} \right]
\begin{array}{c}
    \! R1 + 6R3 \\
    \!\\
    \!
\end{array} \\
\implies & \left[\begin{array}{@{}cccc|c@{}}
    1 & 0 & 0 & -1 & 0\\
    0 & 1 & 0 & -1 & 0\\
    0 & 0 & 1 & -\frac{1}{6}& 0
\end{array} \right]
\end{align*}

The following equations can be formed:
\begin{equation*}
    \begin{array}{cccccc}
        x_1 & & & -x_4 & =  & 0\\
        & x_2 & & -x_4 & = & 0\\
        & & x_3 & -\frac{1}{6}x_4 & = & 0\\
        & & & x_4 & = & x_4\\
    \end{array}
\end{equation*}

Rearranging:
\begin{align*}
    x_1 &= x_4 \\
    x_2 &= x_4 \\
    x_3 &= \frac{1}{6}x_4 \\
    x_4 &= x_4
\end{align*}

\section{Solution for a chemistry class}
An example nonzero solution (since coefficients of compounds coefficients cannot be 0) would be:
\begin{equation*}
    \begin{array}{@{}c@{}}
        x_1 = 6 \\
        x_2 = 6 \\
        x_3 = 1 \\
        x_4 = 6
    \end{array}
\end{equation*}

\section{Solution for a ESC103F class}
The vector form of the solution is given below:
\begin{align*}
    \left[
        \begin{array}{@{}c@{}}
            x_1\\
            x_2\\
            x_3\\
            x_4
        \end{array}
    \right] =
    \left[
        \begin{array}{@{}c@{}}
            0 \\
            0 \\
            0 \\
            0
        \end{array}
    \right] +
    x_4\!\left[
        \begin{array}{@{}c@{}}
            1 \\
            1 \\
            \frac{1}{6} \\
            1
        \end{array}
    \right]
\end{align*}
\end{document}
